\cvsection{Employment}

\begin{cventry}
    {AT\&T}
    {TDP Software Engineer II}
    {Jun. 2022 - Present}
    {Python, Cortex XSOAR, JavaScript, Bash,}{AWS, Azure, Agile, Scrum}
    \begin{cvitems}
        \item Implemented security automation capabilities in AT\&T’s hosted virtual desktops (HVD) to enhance the ability to detect and counteract potential bad actors
        \item Introduced the HVD Reset Integration in Vortex which led to a remarkable 70\% reduction in reset time by seamlessly communicating with the HVD middleware API, streamlining the process of resetting, and erasing sessions
        \item Spearheaded the development of Vortex’s KeyFactor Integration, a product owned by the Cloud Security Enablement’s Encryption team, to manage security encryption certificates on a global scale
        \item Empowered the Encryption team with chain of command escalation automations for expiring certificates, streamlining notifications to certificate owners and Azure Security Groups, resulting in a 65\% improvement in escalation time
        \item Engineered the KeyFactor Integration and its commands in Vortex to communicate with the KeyFactor REST API using Python, significantly enhancing flexibility for playbook development
        \item Optimized AWS and Azure operating expenses by building a new WebPhone Integration using Python. Launching the new WebPhone integration improved connectivity between services and allowed Vortex to retire unused resources in the public cloud
        \item Architected the Vortex NACD API, a partner application in Cloud Security Enablement, to enhance the incident response capabilities for the business
        \item Refactored AWS Lambda functions using Python to enrich executive reporting, Vortex System health, and Integration Health Checks
        \item Launched an internal tool for scanning AT\&T's internal databases for sensitive protection information that stay ahead of the curve by analyzing 3.4 billion data points in 4 hours with 80\% accuracy
        
    \end{cvitems}
\end{cventry}



\begin{cventry}
    {AT\&T}
    {TDP Software Engineer Intern}
    {Jun. to Aug. 2021}
    {Java, Oracle, Bitbucket, Python, HTML/CSS, JavaScript}{Django, Pandas, Agile, Scrum}
    \begin{cvitems}
        \item Devised an automated error detection module that identified anomalies with AT\&T's 5G location-based services, saving the company \$500,000
        % \item Consolidated \$500,000 by creating an automated error detection module that identifies anomalies with AT\&T's 5G location
        % \item Identified specific geo-validation in a database with 10000000 endpoints and aggregated specific use cases   
        \item Engineered a spelling proximity algorithm to correct errors with cell tower information which increased accuracy by 30\% using Java and Oracle
        \item Created HBOConnect, a prototype to engage user personalization, as an integration to HBOMax using Python/Django
    \end{cvitems}
\end{cventry}




% \begin{cventry}
%     {New Jersey Sustainability Reporting Hub}
%     {Software Developer Intern}
%     {Jun. to Aug. 2020}
%     {HTML/CSS, JavaScript, AJAX, GitHub, REST API}{Bootstrap, D3.js}
%     \begin{cvitems}
%         \item Orchestrated and developed a strategic solution, MapSearchUI, to enhance the website experience, which was selected to be implemented for the company
%         \item Refactored the prototype to a front-end application using JavaScript and D3.js, successfully integrating with the company's production stack
%         % \item Improved the site's user experience by integrating an interactive searching tool, MapSearchUI, using JavaScript and D3.js
%         % \item Created a New Jersey image map with links to county-specific archives using HTML
%         \item Utilized the WordPress RESTful API to retrieve meta data  to display a dynamic pie chart of popular tags on the website
%         % \item Enhanced MapSearchUI's user experience by incorporating a mobile view using D3.js to calculate the aspect ratio of the pie chart
%     \end{cvitems}
% \end{cventry}


% \begin{cventry}
%     {TCNJ Tutoring Center}
%     {Computer Science Tutor}
%     {Aug. 2019 to Aug. 2021}
%     {}{}
%     \begin{cvitems}
%         \item Tutored college students in object-oriented programming, data structures, databases, algorithms, and software engineering

%     \end{cvitems}
% \end{cventry}

% \begin{cventry}
%     {Cygnus Professionals}
%     {Software Developer Intern}
%     {Jun. to Aug. 2019}
%     {Python, PostgreSQL, Bitbucket, Jira}{Psycopg2, NumPy, Matplotlib}
%     \begin{cvitems}
%       \item Designed an algorithm that displayed low inventory items in an analytics module of a consumer supplier management product using NumPy and Matplotlib
%       \item Integrated a Python script that dynamically updated the graphs after a database transaction using Matplotlib
%       \item Implemented PostgreSQL queries to populate and graph low inventory items
%     \end{cvitems}
% \end{cventry}

% \begin{cventry}
%     {Acrizen Technologies}
%     {Software Engineer}
%     {June 2016 to Nov. 2017}
%     {Java, GitHub}{}
%     \begin{cvitems}
%         \item Developed an attendance tracking module for a Learning Management System using Java
%         \item Integrated a grading system module using Java switch case
%     \end{cvitems}
% \end{cventry}

\begin{comment}
    \begin{cventry}
        {The College of New Jersey (TCNJ)}
        {Computer Science Department Tutor}
        {Apr. 2018 to May 2019}
        {Java, C++, C, Ruby, PostgreSQL, Regular Expressions}{Rails}
        \begin{cvitems}
            \item Tutoring TCNJ students in computer science coursework concerning topics such as object-oriented programming, data structures, discrete math, algorithms, computer architecture, operating systems, databases, and software engineering.
        \end{cvitems}
    \end{cventry}
\end{comment}

\begin{comment}
    \begin{cventry}
        {Mercer Street Friends}
        {Full Stack Web Developer}
        {Jan. to Feb. 2019}
        {Ruby, PostgreSQL, HTML, SCSS}{Rails, SendGrid}
        \begin{cvitems}
            \item Implemented a novel MVC full stack web application with \textsl{Ruby on Rails} and \textsl{PostgreSQL} for the Trenton-based non-profit organization, Mercer Street Friends, to effectively publicize the positive impacts of its charitable work.
            \item Applied the principles of responsive web design to provide a pleasant user experience on desktop, tablet, and mobile devices.
        \end{cvitems}
    \end{cventry}
\end{comment}
